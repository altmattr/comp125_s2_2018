\usepackage[T1]{fontenc}
\usepackage{pslatex}
\usepackage[pdftex]{color}  
\usepackage[pdftex]{graphicx}     
\usepackage{verbatim}
\usepackage{xcolor}
\usepackage{paralist}
\usepackage[colorlinks=true,urlcolor=red]{hyperref}
\input{codeSnippets.tex}
\input{block.tex}
\usetikzlibrary{arrows.meta}
\usetikzlibrary{arrows,shapes,snakes,automata,backgrounds,petri}
\newcommand*\circled[1]{\tikz[baseline=(char.base)]{
            \node[shape=circle,draw,inner sep=2pt] (char) {#1};}}
            
\setlength{\topmargin}{-0.5in}                  % topmargin now at 1in
\setlength{\textheight}{9.5in}                  % body of text = 9.5in
\setlength{\oddsidemargin}{0in}                 % left margin = 1.0in on odd-numbered pages
\setlength{\evensidemargin}{0in}                % left margin = 1.0in on even-numbered pages 
\setlength{\textwidth}{6.5in}                   % width of text line.
\setlength{\parindent}{0.0in}
\newcommand{\code}{\texttt}

\begin{document}

\definecolor{aquamarine}{rgb}{0,0,0.7}
\definecolor{blue}{rgb}{0,0,0.7}
\definecolor{red}{rgb}{1,0,0}

%
\vspace{0.2in}
\begin{center}
        {\large  %MACQUARIE UNIVERSITY\\
%\medskip
\includegraphics[scale=0.3]{mqlogo.jpeg}\\
\medskip
         {\bf WCOM125/ COMP125 Fundamentals of Computer Science\\
        Workshop - Linked Lists\\}}
\end{center}
\vspace{0.3in}
%

%\renewcommand{\labelenumi}{\arabic{enumi}.}
\renewcommand{\labelenumi}{\alph{enumi}.}
 
\section* {Learning outcomes}

By the end of this session, you will have learnt about linked lists. 

%\begin{center}
%\fbox{
%\begin{minipage}{0.8\textwidth}
%Import project from \texttt{recursionWorkshopTemplate.zip}. 
%\end{minipage}
%}
%\end{center}

\section*{Questions}
\begin{questions}

\section{Node as the primitive for recursive data structure}

All questions in this section use the following definition of \texttt{Node} class

\lstinputlisting[language=Java]{Node.java}

\newpage

\question Draw a memory diagram representing objects in memory after the following code is executed.

\begin{lstlisting}
public class NodeStorage {
	public static void main(String[] args) {
		Node p = new Node(20, null);
		Node g = new Node(50, null);
		Node a = new Node(70, p);
		Node y = new Node(30, null);
		g.setNext(y);
		p.setNext(g);
		y.setData(10);
		p.getNext().getNext().setData(90);
	}
}
\end{lstlisting}

\ifprintanswers
\begin{tikzpicture}
\foreach \x in {0,3,6,9} 
{
	\draw (\x,0) rectangle (\x+2,1);
	\draw (\x+1,0) -- (\x+1,1);
	\ifthenelse{\x=9}
	{
		\node at (\x+1.5, 0.5) {null};
	}
	{
	\draw [->] (\x+1.5, 0.5) -- (\x+3, 0.5);
	}
}
\node at (0.5, 0.5) {70};
\node at (3.5, 0.5) {20};
\node at (6.5, 0.5) {50};
\node at (9.5, 0.5) {10};
\node at (1, -0.5) {\circled{a}};
\node at (4, -0.5) {\circled{p}};
\node at (7, -0.5) {\circled{g}};
\node at (10, -0.5) {\circled{y}};
\end{tikzpicture}
\else
\fi

\question Consider the following code:

\begin{lstlisting}
Node e = new Node(10, null);
Node d = new Node(20, e);
Node c = new Node(90, d);
Node b = new Node(60, c);
Node a = new Node(30, b);
\end{lstlisting}

\ifprintanswers
\begin{tikzpicture}
\def\data{{30,60,90,20,10}}
\foreach \x in {0,3,6,9,12} 
{
	\draw (\x,0) rectangle (\x+2,1);
	\draw (\x+1,0) -- (\x+1,1);
	\ifthenelse{\x=12}
	{
		\node at (\x+1.5, 0.5) {null};
	}
	{
	\draw [->] (\x+1.5, 0.5) -- (\x+3, 0.5);
	}
}
\node at (0.5, 0.5) {30};
\node at (3.5, 0.5) {60};
\node at (6.5, 0.5) {90};
\node at (9.5, 0.5) {20};
\node at (12.5, 0.5) {10};

\node at (1, -0.5) {\circled{a}};
\node at (4, -0.5) {\circled{b}};
\node at (7, -0.5) {\circled{c}};
\node at (10, -0.5) {\circled{d}};
\node at (13, -0.5) {\circled{e}};
\end{tikzpicture}
\else
\fi

Write a piece of code that displays the data value of each node, starting at \texttt{Node a}. You must use a loop to do this.

\question Using the same definition for class \texttt{Node} as the previous question, what is the output produced by the following piece of code? 

\begin{lstlisting}
public class Client {
	public static void main(String[] args) {
		Node n = new Node(1, null);
		for(int i=1; i < 4; i++) {
			Node temp = new Node(2*i+1, n);	
			n = temp;
		}
		
		Node current = n;
		while(current != null) {
			System.out.println(current.getData());
			current = current.getNext();
		}
	}
}
\end{lstlisting}

\ifprintanswers
\newpage
SOLUTION: Output is -
\begin{lstlisting}
7
5
3
1
\end{lstlisting}
\else
\newpage
\fi

\question Consider the following class:

\begin{lstlisting}
class Node2 {
	private int data;
	private Node2 a, b, c;
	public Node2(int d, Node _a, Node _b, Node _c) {
		data = d;
		a = _a;
		b = _b;
		c = _c;
	}
}

public class Client {
	public static void main(String[] args) {
		Node2 n1 = new Node(20, null, null, null);
		Node2 n2 = new Node(50, n1, null, null);
		Node2 n3 = new Node(10, n2, n1, null);
		Node2 n4 = new Node(70, n3, n2, n1);
	}
}
\end{lstlisting}

Draw a graph illustrating the nodes and the links between them. Provide a direction and label for each link.

\ifprintanswers
\begin{tikzpicture}
\draw (0, 4) [text width = 1.25cm] rectangle node{n1 (a, b, c = null)} (2,6); 
\draw (4, 6) [text width = 1.25cm] rectangle node{n2 (b, c = null)} (6,8); 
\draw (4, 2) [text width = 1.25cm] rectangle node{n3 (c = null)} (6,4); 
\draw (8, 4) [text width = 1cm] rectangle node{n4} (10,6); 
\draw[->](4,7)--(2,5.5) node[near start,above]{a};
\draw[->](5,4)--(5,6) node[near start,above]{a};
\draw[->](4,3)--(2,4.5) node[near start,left]{b};
\draw[->](8, 5)--(2, 5) node[near start,above]{c};
\draw[->](8, 5.5)--(6, 7) node[near start,above]{b};
\draw[->](8, 4.5)--(6, 3) node[near start,above]{a};
\end{tikzpicture}
\else
\fi

\section{Java's built-in LinkedList class}

Java has a built-in implementation of linked lists in class \texttt{LinkedList}. It behaves \textbf{\emph{almost}} identically to \texttt{ArrayList} class. Thus, for the user, there is hardly any difference. 

\question Write a method \texttt{countPositives} that when passed an \texttt{LinkedList} of \texttt{Double} objects, returns the number of positive items in the \texttt{LinkedList}. The method should return 0 if the list is \texttt{null} or empty.

\begin{lstlisting}
	public static int countPositives(LinkedList <Double> list)
\end{lstlisting}

\begin{solution}
\begin{lstlisting}
public static int countPositives(LinkedList <Double> list) {
		if(list == null || list.size() == 0) 
			return 0;
		int result = 0;
		for(Double item: list)
			if(item > 0)
				result++;
		return result;
	}	
\end{lstlisting}	
\end{solution}


\question Write a method \texttt{countMatches} that when passed an \texttt{LinkedList} of \texttt{String} objects and a String \texttt{target}, returns the number of items in the list that contains \texttt{target}. The method should return 0 if the list is \texttt{null} or empty. For example, if \texttt{list = ["thereby", "they", "proved", "the", "other", "guy", "was", "the", "father"]} and \texttt{target = "the"}, the method should return 6, as there are six Strings containing \texttt{"the"}.

\begin{lstlisting}
	public static int countMatches(LinkedList<String> list, String target)
\end{lstlisting}

\begin{solution}
\begin{lstlisting}
	public static int countMatches(LinkedList<String> list, String target) {
		if(list == null || list.size() == 0) 
			return 0;
		int result = 0;
		for(String item: list)
			if(item.contains(target))			
				result++;
		return result;
	}	
\end{lstlisting}	
\end{solution}

\question Add a method \texttt{count} that when passed an \texttt{LinkedList<Integer> list}, returns the number of prime numbers in \texttt{list}.

\begin{lstlisting}
	public static int countPrimes(LinkedList<Integer> list)
\end{lstlisting}

\begin{solution}
\begin{lstlisting}
public static int countPrimes(LinkedList<Integer> list) {
	if(list == null)
		return 0;
	int result = 0;
	for(Integer item: list)
		if(isPrime(item))
			result++;
	return result;
}

public static boolean isPrime(int n) {
	if(n < 2)
		return false;
	for(int i=2; i*i <= n; i++)
		if(n%i == 0)
			return false;
	return true;
}
\end{lstlisting}
\end{solution}

\question \textbf{(D-level)} Write a method that when passed a \texttt{LinkedList} of integers, returns a number constructed with the first digit of each item of the list. The method should return 0 if the list is \texttt{null} or empty. For example, if the list is [15, 673, 8914], the method returns the number 168.

\begin{solution}
\begin{lstlisting}
public static int getFirstDigitNumber(LinkedList <Character> list) {
	if(list == null) 
		return null;
	int result = 0;
	for(Integer item: list)
		result = result*10 + firstDigit(item);
	return result;
}	

public static int firstDigit(int n) {
	if(n == 0)
		return 0;
		
	if(n < 1)
		n*=-1;
	
	while(n > 10) {
		n/=10;
	}
	return n;
}
\end{lstlisting}	
\end{solution}


\question  Write a method \texttt{getPerfectSquares} that when passed a \texttt{LinkedList<Integer> list}, returns a list containing perfect squares (squares of integers) in that list.

\begin{lstlisting}
public static LinkedList<Integer> getPerfectSquares(LinkedList<Integer> list)
\end{lstlisting}

\begin{solution}
\begin{lstlisting}
public static LinkedList<Integer> getPerfectSquares(LinkedList<Integer> list) {	if(list == null)
		return null;

	LinkedList<Integer> result = new LinkedList<Integer>();
	for(Integer item: list) {
		double root = Math.sqrt(item * 1.0);
		if((int)root * root == item)
			result.add(item);
	}
	return result;
}
\end{lstlisting}
\end{solution}
\end{questions}
	
\end{document}

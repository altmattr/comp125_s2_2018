\usepackage[T1]{fontenc}
\usepackage{pslatex}
 \usepackage[pdftex]{color}  
 \usepackage[pdftex]{graphicx}     
\usepackage{verbatim}
\newenvironment{myverbatim}{\color{aquamarine}\verbatim}{\endverbatim}
% \usepackage[dvips]{color}  
% \usepackage[dvips]{graphicx}     
\usepackage{amsmath} 
\usepackage{amssymb} 
\usepackage{amsfonts} 
\usepackage{amsthm}
\usepackage{tagging}
\usepackage[colorlinks=true,urlcolor=red]{hyperref}
\setlength{\topmargin}{-0.5in}                  % topmargin now at 1in
\setlength{\textheight}{9.5in}                  % body of text = 9.5in
\usepackage[T1]{fontenc}
\usepackage{pslatex}
\usepackage[pdftex]{color}  
\usepackage[pdftex]{graphicx}     
\usepackage{verbatim}
\usepackage{xcolor}
\usepackage{paralist}
\usepackage{tagging}
\usepackage{hyperref}

\usepackage[colorlinks=true,urlcolor=red]{hyperref}
\setlength{\topmargin}{-0.5in}                  % topmargin now at 1in
\setlength{\textheight}{9.5in}                  % body of text = 9.5in
\setlength{\oddsidemargin}{0in}                 % left margin = 1.0in on odd-numbered pages
\setlength{\evensidemargin}{0in}                % left margin = 1.0in on even-numbered pages 
\setlength{\textwidth}{6.5in}                   % width of text line.
\setlength{\parindent}{0.0in}
\newcommand{\code}{\texttt}

\usepackage{listings}
\lstset{%
	language=Java,
	basicstyle=\footnotesize\ttfamily,
	numbers=left,
	numberstyle=\tiny,        
	xleftmargin=17pt,
        	xrightmargin=5pt,
	frame=single,
	breaklines=true,
	moredelim=**[is][\color{red}]{@}{@}
}

\lstdefinestyle{buggy}{
  language=Java,
  emptylines=1,
  breaklines=true,
  basicstyle=\ttfamily\color{black},
  moredelim=**[is][\color{red}]{@}{@},
}

\lstdefinestyle{correct}{
  language=Java,
  emptylines=1,
  breaklines=true,
  basicstyle=\ttfamily\color{black},
  moredelim=**[is][\color{blue}]{@}{@},
}


\begin{document}

\definecolor{aquamarine}{rgb}{0,0,0.7}
\definecolor{blue}{rgb}{0,0,0.7}
\definecolor{red}{rgb}{1,0,0}

%
\vspace{0.2in}
\begin{center}
        {\large  %MACQUARIE UNIVERSITY\\
%\medskip
\includegraphics[scale=0.3]{../../logo.jpg}\\
\medskip
        {\it  Faculty of Science and Engineering\\}
        \vspace{0.2in}
         {\bf COMP125 Fundamentals of Computer Science\\
        Workshop Week 5\\}}
\end{center}
\vspace{0.3in}
%

%\renewcommand{\labelenumi}{\arabic{enumi}.}
\renewcommand{\labelenumi}{\alph{enumi}.}
 
\section*{Learning outcomes}
Following are this week's learning outcomes,
\begin{enumerate}
\item Experiment with regards to time complexity
\item Write methods and test them using JUnit tests
\item Correct someone else's buggy code using JUnit tests
\item Write JUnit tests for given methods (that may or may not be buggy)
\end{enumerate}

\begin{center}
\fbox{\fbox{\parbox{0.8\textwidth}{\centering
Import the project from archive file timeComplexityJUnit.zip}}}
\end{center}

\section*{Questions}
\begin{questions}

\question \textbf{(Assessed)}
Correct the following methods in class \texttt{Fraction} based on tests in \texttt{TestFraction},

\begin{enumerate}
	\item multiply
	\item equals
\end{enumerate}

\question \textbf{(Assessed)}
Complete the following test methods in class \texttt{TestFraction},

\begin{enumerate}
	\item testAdd
	\item testSubtract
\end{enumerate}

Which method, \texttt{add} or \texttt{subtract}, in class \texttt{Fraction} has a bug? Remove the bug.

\section*{Experiments with time complexity}

\question 
Compare the running times in \texttt{TimeComplexityClient} for two versions for finding the index of the first occurrence of an item in an array. The methods, in \texttt{TimeComplexityService} are,

\begin{enumerate}
\item inefficientSearch
\item efficientSearch	
\end{enumerate}

Go through the two methods and determine why the method that is more efficient is so.

\question
Repeat the previous exercise for \texttt{sumVersion1} vs. \texttt{sumVersion2}.

\question \textbf{(Assessed)}
Improve the efficiency of the method \texttt{meFailEnglishThatsUnpossible}.

\question
Write down the time complexities in Big-O notation for the methods,

\begin{enumerate}
	\item foo1
	\item foo2
	\item foo3
\end{enumerate}

\end{questions}
\end{document}
